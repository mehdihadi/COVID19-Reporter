% Options for packages loaded elsewhere
\PassOptionsToPackage{unicode}{hyperref}
\PassOptionsToPackage{hyphens}{url}
%
\documentclass[
]{article}
\usepackage{lmodern}
\usepackage{amssymb,amsmath}
\usepackage{ifxetex,ifluatex}
\ifnum 0\ifxetex 1\fi\ifluatex 1\fi=0 % if pdftex
  \usepackage[T1]{fontenc}
  \usepackage[utf8]{inputenc}
  \usepackage{textcomp} % provide euro and other symbols
\else % if luatex or xetex
  \usepackage{unicode-math}
  \defaultfontfeatures{Scale=MatchLowercase}
  \defaultfontfeatures[\rmfamily]{Ligatures=TeX,Scale=1}
\fi
% Use upquote if available, for straight quotes in verbatim environments
\IfFileExists{upquote.sty}{\usepackage{upquote}}{}
\IfFileExists{microtype.sty}{% use microtype if available
  \usepackage[]{microtype}
  \UseMicrotypeSet[protrusion]{basicmath} % disable protrusion for tt fonts
}{}
\makeatletter
\@ifundefined{KOMAClassName}{% if non-KOMA class
  \IfFileExists{parskip.sty}{%
    \usepackage{parskip}
  }{% else
    \setlength{\parindent}{0pt}
    \setlength{\parskip}{6pt plus 2pt minus 1pt}}
}{% if KOMA class
  \KOMAoptions{parskip=half}}
\makeatother
\usepackage{xcolor}
\IfFileExists{xurl.sty}{\usepackage{xurl}}{} % add URL line breaks if available
\IfFileExists{bookmark.sty}{\usepackage{bookmark}}{\usepackage{hyperref}}
\hypersetup{
  pdftitle={Untitled},
  pdfauthor={hadi},
  hidelinks,
  pdfcreator={LaTeX via pandoc}}
\urlstyle{same} % disable monospaced font for URLs
\usepackage[margin=1in]{geometry}
\usepackage{graphicx,grffile}
\makeatletter
\def\maxwidth{\ifdim\Gin@nat@width>\linewidth\linewidth\else\Gin@nat@width\fi}
\def\maxheight{\ifdim\Gin@nat@height>\textheight\textheight\else\Gin@nat@height\fi}
\makeatother
% Scale images if necessary, so that they will not overflow the page
% margins by default, and it is still possible to overwrite the defaults
% using explicit options in \includegraphics[width, height, ...]{}
\setkeys{Gin}{width=\maxwidth,height=\maxheight,keepaspectratio}
% Set default figure placement to htbp
\makeatletter
\def\fps@figure{htbp}
\makeatother
\setlength{\emergencystretch}{3em} % prevent overfull lines
\providecommand{\tightlist}{%
  \setlength{\itemsep}{0pt}\setlength{\parskip}{0pt}}
\setcounter{secnumdepth}{-\maxdimen} % remove section numbering

\title{Untitled}
\author{hadi}
\date{4/11/2020}

\begin{document}
\maketitle

\hypertarget{introduction}{%
\subsubsection{Introduction}\label{introduction}}

Coronavirus disease 2019 (COVID-19) is an infectious disease caused by
severe acute respiratory syndrome coronavirus 2 (SARS-CoV-2)
\href{https://www.who.int/emergencies/diseases/novel-coronavirus-2019/technical-guidance/naming-the-coronavirus-disease-\%28covid-2019\%29-and-the-virus-that-causes-it}{(1)}.
On 31 December 2019, the WHO China Country Office was informed the first
cases of Coronavirus disease 2019 (COVID-19) in Wuhan City, Hubei
Province of China
\href{https://www.who.int/docs/default-source/coronaviruse/situation-reports/20200121-sitrep-1-2019-ncov.pdf?sfvrsn=20a99c10_4}{(2)}.

COVID-19 disease is spreading rapidly throughout the world. The outbreak
was declared a public health emergency of international concern on 30
January 2020
\href{https://www.who.int/emergencies/diseases/novel-coronavirus-2019/events-as-they-happen}{(3)}.WHO
declared the outbreak a global pandemic on March 11, 2020
\href{https://www.who.int/dg/speeches/detail/who-director-general-s-opening-remarks-at-the-media-briefing-on-covid-19---11-march-2020}{(4)}.
In response to this ongoing public health emergency, the Center for
Systems Science and Engineering \href{https://systems.jhu.edu/}{(CSSE)}
at Johns Hopkins University, Baltimore, MD, USA, developed an online
interactive \href{https://coronavirus.jhu.edu/map.html}{dashboard}, to
visualize and track reported cases of COVID-19 in real time
\href{https://www.thelancet.com/journals/laninf/article/PIIS1473-3099\%2820\%2930120-1/fulltext}{(5)}.
They developed the dashboard to provide researchers, public health
authorities, and the general public with a tool to track the outbreak.
Johns Hopkins University was collected COVID-19 outbreak data from
several sources including including
\href{https://www.who.int/emergencies/diseases/novel-coronavirus-2019/situation-reports}{WHO},
U.S. \href{https://www.cdc.gov/coronavirus/2019-ncov/index.html}{CDC},
\href{https://www.ecdc.europa.eu/en/home}{ECDC}, China CDC
\href{http://www.chinacdc.cn/en/}{(CCDC)},
\href{http://www.nhc.gov.cn/yjb/s3578/new_list.shtml}{NHC} and
\href{https://3g.dxy.cn/newh5/view/pneumonia?scene=2\&clicktime=1579582238\&enterid=1579582238\&from=singlemessage\&isappinstalled=0}{DXY},
as well as city-level and state-level health authorities. They made all
collected data freely available, through a
\href{https://github.com/CSSEGISandData/COVID-19}{GitHub repository} to
be used for further analyzing by researchers. However, the data in this
repository requires pre-processing or more aggregating before any use,
which I thought is out of the boredom of many researchers. In addition,
Johns Hopkins dashboard does not allow options like downloading of
processed data, analyzing of date at provincial-scale, or allowing the
use of epidemiological models such as SEIR to be applied to country data
for estimation of epidemiological parameters. Therefore, developing a
dashboard with more capabilities to assist the researchers in analysing
the COVID-19 pandemic situation for countries or provinces is necessary.

\hypertarget{what-this-dashboard-can-do}{%
\subsubsection{What this dashboard can
do?}\label{what-this-dashboard-can-do}}

In this new dashboard the pandemic data was directly called from GitHub
repository of Johns Hopkins University and following capabilities are
provided:

\begin{itemize}
\tightlist
\item
  Producing up-to-date charts of cumulative or daily trends of epidemic
  for world, countries and provinces
\item
  Producing up-to-date charts of death rate for world, countries and
  provinces

  \begin{itemize}
  \tightlist
  \item
    Death rate was determined based on
    \href{https://www.thelancet.com/journals/laninf/article/PIIS1473-3099\%2820\%2930195-X/fulltext}{The
    Lancet article} by dividing the number of deaths on a given day by
    the number of patients with confirmed COVID-19 infection 14 days
    before.
  \end{itemize}
\item
  Possibility to download aggregated datasets by world, countries and
  provinces
\item
  Creating a dynamic map to show the trend of COVID-19 by confirmed,
  recovered and deaths cases
\item
  Ranking the countries in terms of cumulative and daily data by
  confirmed, recovered and deaths cases
\item
  Formulation of a simple version of an SEIR model to reflect the
  disease dynamics
\item
  Estimation of epidemiological parameters including infection rate
  (\(\beta\)), incubation rate (\(\sigma\)), recovery rate (\(\gamma\))
  and reproduction number (\(R_0\))
\item
  Assessing the potential effect of social distancing intervention on
  COVID-19 spread using SEIR model
\end{itemize}

Drawing of my 5-year-old daughter

\end{document}
